
% Default to the notebook output style

    


% Inherit from the specified cell style.




    
\documentclass[11pt]{article}

    
    
    \usepackage[T1]{fontenc}
    % Nicer default font (+ math font) than Computer Modern for most use cases
    \usepackage{mathpazo}

    % Basic figure setup, for now with no caption control since it's done
    % automatically by Pandoc (which extracts ![](path) syntax from Markdown).
    \usepackage{graphicx}
    % We will generate all images so they have a width \maxwidth. This means
    % that they will get their normal width if they fit onto the page, but
    % are scaled down if they would overflow the margins.
    \makeatletter
    \def\maxwidth{\ifdim\Gin@nat@width>\linewidth\linewidth
    \else\Gin@nat@width\fi}
    \makeatother
    \let\Oldincludegraphics\includegraphics
    % Set max figure width to be 80% of text width, for now hardcoded.
    \renewcommand{\includegraphics}[1]{\Oldincludegraphics[width=.8\maxwidth]{#1}}
    % Ensure that by default, figures have no caption (until we provide a
    % proper Figure object with a Caption API and a way to capture that
    % in the conversion process - todo).
    \usepackage{caption}
    \DeclareCaptionLabelFormat{nolabel}{}
    \captionsetup{labelformat=nolabel}

    \usepackage{adjustbox} % Used to constrain images to a maximum size 
    \usepackage{xcolor} % Allow colors to be defined
    \usepackage{enumerate} % Needed for markdown enumerations to work
    \usepackage{geometry} % Used to adjust the document margins
    \usepackage{amsmath} % Equations
    \usepackage{amssymb} % Equations
    \usepackage{textcomp} % defines textquotesingle
    % Hack from http://tex.stackexchange.com/a/47451/13684:
    \AtBeginDocument{%
        \def\PYZsq{\textquotesingle}% Upright quotes in Pygmentized code
    }
    \usepackage{upquote} % Upright quotes for verbatim code
    \usepackage{eurosym} % defines \euro
    \usepackage[mathletters]{ucs} % Extended unicode (utf-8) support
    \usepackage[utf8x]{inputenc} % Allow utf-8 characters in the tex document
    \usepackage{fancyvrb} % verbatim replacement that allows latex
    \usepackage{grffile} % extends the file name processing of package graphics 
                         % to support a larger range 
    % The hyperref package gives us a pdf with properly built
    % internal navigation ('pdf bookmarks' for the table of contents,
    % internal cross-reference links, web links for URLs, etc.)
    \usepackage{hyperref}
    \usepackage{longtable} % longtable support required by pandoc >1.10
    \usepackage{booktabs}  % table support for pandoc > 1.12.2
    \usepackage[inline]{enumitem} % IRkernel/repr support (it uses the enumerate* environment)
    \usepackage[normalem]{ulem} % ulem is needed to support strikethroughs (\sout)
                                % normalem makes italics be italics, not underlines
    

    
    
    % Colors for the hyperref package
    \definecolor{urlcolor}{rgb}{0,.145,.698}
    \definecolor{linkcolor}{rgb}{.71,0.21,0.01}
    \definecolor{citecolor}{rgb}{.12,.54,.11}

    % ANSI colors
    \definecolor{ansi-black}{HTML}{3E424D}
    \definecolor{ansi-black-intense}{HTML}{282C36}
    \definecolor{ansi-red}{HTML}{E75C58}
    \definecolor{ansi-red-intense}{HTML}{B22B31}
    \definecolor{ansi-green}{HTML}{00A250}
    \definecolor{ansi-green-intense}{HTML}{007427}
    \definecolor{ansi-yellow}{HTML}{DDB62B}
    \definecolor{ansi-yellow-intense}{HTML}{B27D12}
    \definecolor{ansi-blue}{HTML}{208FFB}
    \definecolor{ansi-blue-intense}{HTML}{0065CA}
    \definecolor{ansi-magenta}{HTML}{D160C4}
    \definecolor{ansi-magenta-intense}{HTML}{A03196}
    \definecolor{ansi-cyan}{HTML}{60C6C8}
    \definecolor{ansi-cyan-intense}{HTML}{258F8F}
    \definecolor{ansi-white}{HTML}{C5C1B4}
    \definecolor{ansi-white-intense}{HTML}{A1A6B2}

    % commands and environments needed by pandoc snippets
    % extracted from the output of `pandoc -s`
    \providecommand{\tightlist}{%
      \setlength{\itemsep}{0pt}\setlength{\parskip}{0pt}}
    \DefineVerbatimEnvironment{Highlighting}{Verbatim}{commandchars=\\\{\}}
    % Add ',fontsize=\small' for more characters per line
    \newenvironment{Shaded}{}{}
    \newcommand{\KeywordTok}[1]{\textcolor[rgb]{0.00,0.44,0.13}{\textbf{{#1}}}}
    \newcommand{\DataTypeTok}[1]{\textcolor[rgb]{0.56,0.13,0.00}{{#1}}}
    \newcommand{\DecValTok}[1]{\textcolor[rgb]{0.25,0.63,0.44}{{#1}}}
    \newcommand{\BaseNTok}[1]{\textcolor[rgb]{0.25,0.63,0.44}{{#1}}}
    \newcommand{\FloatTok}[1]{\textcolor[rgb]{0.25,0.63,0.44}{{#1}}}
    \newcommand{\CharTok}[1]{\textcolor[rgb]{0.25,0.44,0.63}{{#1}}}
    \newcommand{\StringTok}[1]{\textcolor[rgb]{0.25,0.44,0.63}{{#1}}}
    \newcommand{\CommentTok}[1]{\textcolor[rgb]{0.38,0.63,0.69}{\textit{{#1}}}}
    \newcommand{\OtherTok}[1]{\textcolor[rgb]{0.00,0.44,0.13}{{#1}}}
    \newcommand{\AlertTok}[1]{\textcolor[rgb]{1.00,0.00,0.00}{\textbf{{#1}}}}
    \newcommand{\FunctionTok}[1]{\textcolor[rgb]{0.02,0.16,0.49}{{#1}}}
    \newcommand{\RegionMarkerTok}[1]{{#1}}
    \newcommand{\ErrorTok}[1]{\textcolor[rgb]{1.00,0.00,0.00}{\textbf{{#1}}}}
    \newcommand{\NormalTok}[1]{{#1}}
    
    % Additional commands for more recent versions of Pandoc
    \newcommand{\ConstantTok}[1]{\textcolor[rgb]{0.53,0.00,0.00}{{#1}}}
    \newcommand{\SpecialCharTok}[1]{\textcolor[rgb]{0.25,0.44,0.63}{{#1}}}
    \newcommand{\VerbatimStringTok}[1]{\textcolor[rgb]{0.25,0.44,0.63}{{#1}}}
    \newcommand{\SpecialStringTok}[1]{\textcolor[rgb]{0.73,0.40,0.53}{{#1}}}
    \newcommand{\ImportTok}[1]{{#1}}
    \newcommand{\DocumentationTok}[1]{\textcolor[rgb]{0.73,0.13,0.13}{\textit{{#1}}}}
    \newcommand{\AnnotationTok}[1]{\textcolor[rgb]{0.38,0.63,0.69}{\textbf{\textit{{#1}}}}}
    \newcommand{\CommentVarTok}[1]{\textcolor[rgb]{0.38,0.63,0.69}{\textbf{\textit{{#1}}}}}
    \newcommand{\VariableTok}[1]{\textcolor[rgb]{0.10,0.09,0.49}{{#1}}}
    \newcommand{\ControlFlowTok}[1]{\textcolor[rgb]{0.00,0.44,0.13}{\textbf{{#1}}}}
    \newcommand{\OperatorTok}[1]{\textcolor[rgb]{0.40,0.40,0.40}{{#1}}}
    \newcommand{\BuiltInTok}[1]{{#1}}
    \newcommand{\ExtensionTok}[1]{{#1}}
    \newcommand{\PreprocessorTok}[1]{\textcolor[rgb]{0.74,0.48,0.00}{{#1}}}
    \newcommand{\AttributeTok}[1]{\textcolor[rgb]{0.49,0.56,0.16}{{#1}}}
    \newcommand{\InformationTok}[1]{\textcolor[rgb]{0.38,0.63,0.69}{\textbf{\textit{{#1}}}}}
    \newcommand{\WarningTok}[1]{\textcolor[rgb]{0.38,0.63,0.69}{\textbf{\textit{{#1}}}}}
    
    
    % Define a nice break command that doesn't care if a line doesn't already
    % exist.
    \def\br{\hspace*{\fill} \\* }
    % Math Jax compatability definitions
    \def\gt{>}
    \def\lt{<}
    % Document parameters
    \title{FINAL\_Assignment 1}
    
    
    

    % Pygments definitions
    
\makeatletter
\def\PY@reset{\let\PY@it=\relax \let\PY@bf=\relax%
    \let\PY@ul=\relax \let\PY@tc=\relax%
    \let\PY@bc=\relax \let\PY@ff=\relax}
\def\PY@tok#1{\csname PY@tok@#1\endcsname}
\def\PY@toks#1+{\ifx\relax#1\empty\else%
    \PY@tok{#1}\expandafter\PY@toks\fi}
\def\PY@do#1{\PY@bc{\PY@tc{\PY@ul{%
    \PY@it{\PY@bf{\PY@ff{#1}}}}}}}
\def\PY#1#2{\PY@reset\PY@toks#1+\relax+\PY@do{#2}}

\expandafter\def\csname PY@tok@w\endcsname{\def\PY@tc##1{\textcolor[rgb]{0.73,0.73,0.73}{##1}}}
\expandafter\def\csname PY@tok@c\endcsname{\let\PY@it=\textit\def\PY@tc##1{\textcolor[rgb]{0.25,0.50,0.50}{##1}}}
\expandafter\def\csname PY@tok@cp\endcsname{\def\PY@tc##1{\textcolor[rgb]{0.74,0.48,0.00}{##1}}}
\expandafter\def\csname PY@tok@k\endcsname{\let\PY@bf=\textbf\def\PY@tc##1{\textcolor[rgb]{0.00,0.50,0.00}{##1}}}
\expandafter\def\csname PY@tok@kp\endcsname{\def\PY@tc##1{\textcolor[rgb]{0.00,0.50,0.00}{##1}}}
\expandafter\def\csname PY@tok@kt\endcsname{\def\PY@tc##1{\textcolor[rgb]{0.69,0.00,0.25}{##1}}}
\expandafter\def\csname PY@tok@o\endcsname{\def\PY@tc##1{\textcolor[rgb]{0.40,0.40,0.40}{##1}}}
\expandafter\def\csname PY@tok@ow\endcsname{\let\PY@bf=\textbf\def\PY@tc##1{\textcolor[rgb]{0.67,0.13,1.00}{##1}}}
\expandafter\def\csname PY@tok@nb\endcsname{\def\PY@tc##1{\textcolor[rgb]{0.00,0.50,0.00}{##1}}}
\expandafter\def\csname PY@tok@nf\endcsname{\def\PY@tc##1{\textcolor[rgb]{0.00,0.00,1.00}{##1}}}
\expandafter\def\csname PY@tok@nc\endcsname{\let\PY@bf=\textbf\def\PY@tc##1{\textcolor[rgb]{0.00,0.00,1.00}{##1}}}
\expandafter\def\csname PY@tok@nn\endcsname{\let\PY@bf=\textbf\def\PY@tc##1{\textcolor[rgb]{0.00,0.00,1.00}{##1}}}
\expandafter\def\csname PY@tok@ne\endcsname{\let\PY@bf=\textbf\def\PY@tc##1{\textcolor[rgb]{0.82,0.25,0.23}{##1}}}
\expandafter\def\csname PY@tok@nv\endcsname{\def\PY@tc##1{\textcolor[rgb]{0.10,0.09,0.49}{##1}}}
\expandafter\def\csname PY@tok@no\endcsname{\def\PY@tc##1{\textcolor[rgb]{0.53,0.00,0.00}{##1}}}
\expandafter\def\csname PY@tok@nl\endcsname{\def\PY@tc##1{\textcolor[rgb]{0.63,0.63,0.00}{##1}}}
\expandafter\def\csname PY@tok@ni\endcsname{\let\PY@bf=\textbf\def\PY@tc##1{\textcolor[rgb]{0.60,0.60,0.60}{##1}}}
\expandafter\def\csname PY@tok@na\endcsname{\def\PY@tc##1{\textcolor[rgb]{0.49,0.56,0.16}{##1}}}
\expandafter\def\csname PY@tok@nt\endcsname{\let\PY@bf=\textbf\def\PY@tc##1{\textcolor[rgb]{0.00,0.50,0.00}{##1}}}
\expandafter\def\csname PY@tok@nd\endcsname{\def\PY@tc##1{\textcolor[rgb]{0.67,0.13,1.00}{##1}}}
\expandafter\def\csname PY@tok@s\endcsname{\def\PY@tc##1{\textcolor[rgb]{0.73,0.13,0.13}{##1}}}
\expandafter\def\csname PY@tok@sd\endcsname{\let\PY@it=\textit\def\PY@tc##1{\textcolor[rgb]{0.73,0.13,0.13}{##1}}}
\expandafter\def\csname PY@tok@si\endcsname{\let\PY@bf=\textbf\def\PY@tc##1{\textcolor[rgb]{0.73,0.40,0.53}{##1}}}
\expandafter\def\csname PY@tok@se\endcsname{\let\PY@bf=\textbf\def\PY@tc##1{\textcolor[rgb]{0.73,0.40,0.13}{##1}}}
\expandafter\def\csname PY@tok@sr\endcsname{\def\PY@tc##1{\textcolor[rgb]{0.73,0.40,0.53}{##1}}}
\expandafter\def\csname PY@tok@ss\endcsname{\def\PY@tc##1{\textcolor[rgb]{0.10,0.09,0.49}{##1}}}
\expandafter\def\csname PY@tok@sx\endcsname{\def\PY@tc##1{\textcolor[rgb]{0.00,0.50,0.00}{##1}}}
\expandafter\def\csname PY@tok@m\endcsname{\def\PY@tc##1{\textcolor[rgb]{0.40,0.40,0.40}{##1}}}
\expandafter\def\csname PY@tok@gh\endcsname{\let\PY@bf=\textbf\def\PY@tc##1{\textcolor[rgb]{0.00,0.00,0.50}{##1}}}
\expandafter\def\csname PY@tok@gu\endcsname{\let\PY@bf=\textbf\def\PY@tc##1{\textcolor[rgb]{0.50,0.00,0.50}{##1}}}
\expandafter\def\csname PY@tok@gd\endcsname{\def\PY@tc##1{\textcolor[rgb]{0.63,0.00,0.00}{##1}}}
\expandafter\def\csname PY@tok@gi\endcsname{\def\PY@tc##1{\textcolor[rgb]{0.00,0.63,0.00}{##1}}}
\expandafter\def\csname PY@tok@gr\endcsname{\def\PY@tc##1{\textcolor[rgb]{1.00,0.00,0.00}{##1}}}
\expandafter\def\csname PY@tok@ge\endcsname{\let\PY@it=\textit}
\expandafter\def\csname PY@tok@gs\endcsname{\let\PY@bf=\textbf}
\expandafter\def\csname PY@tok@gp\endcsname{\let\PY@bf=\textbf\def\PY@tc##1{\textcolor[rgb]{0.00,0.00,0.50}{##1}}}
\expandafter\def\csname PY@tok@go\endcsname{\def\PY@tc##1{\textcolor[rgb]{0.53,0.53,0.53}{##1}}}
\expandafter\def\csname PY@tok@gt\endcsname{\def\PY@tc##1{\textcolor[rgb]{0.00,0.27,0.87}{##1}}}
\expandafter\def\csname PY@tok@err\endcsname{\def\PY@bc##1{\setlength{\fboxsep}{0pt}\fcolorbox[rgb]{1.00,0.00,0.00}{1,1,1}{\strut ##1}}}
\expandafter\def\csname PY@tok@kc\endcsname{\let\PY@bf=\textbf\def\PY@tc##1{\textcolor[rgb]{0.00,0.50,0.00}{##1}}}
\expandafter\def\csname PY@tok@kd\endcsname{\let\PY@bf=\textbf\def\PY@tc##1{\textcolor[rgb]{0.00,0.50,0.00}{##1}}}
\expandafter\def\csname PY@tok@kn\endcsname{\let\PY@bf=\textbf\def\PY@tc##1{\textcolor[rgb]{0.00,0.50,0.00}{##1}}}
\expandafter\def\csname PY@tok@kr\endcsname{\let\PY@bf=\textbf\def\PY@tc##1{\textcolor[rgb]{0.00,0.50,0.00}{##1}}}
\expandafter\def\csname PY@tok@bp\endcsname{\def\PY@tc##1{\textcolor[rgb]{0.00,0.50,0.00}{##1}}}
\expandafter\def\csname PY@tok@fm\endcsname{\def\PY@tc##1{\textcolor[rgb]{0.00,0.00,1.00}{##1}}}
\expandafter\def\csname PY@tok@vc\endcsname{\def\PY@tc##1{\textcolor[rgb]{0.10,0.09,0.49}{##1}}}
\expandafter\def\csname PY@tok@vg\endcsname{\def\PY@tc##1{\textcolor[rgb]{0.10,0.09,0.49}{##1}}}
\expandafter\def\csname PY@tok@vi\endcsname{\def\PY@tc##1{\textcolor[rgb]{0.10,0.09,0.49}{##1}}}
\expandafter\def\csname PY@tok@vm\endcsname{\def\PY@tc##1{\textcolor[rgb]{0.10,0.09,0.49}{##1}}}
\expandafter\def\csname PY@tok@sa\endcsname{\def\PY@tc##1{\textcolor[rgb]{0.73,0.13,0.13}{##1}}}
\expandafter\def\csname PY@tok@sb\endcsname{\def\PY@tc##1{\textcolor[rgb]{0.73,0.13,0.13}{##1}}}
\expandafter\def\csname PY@tok@sc\endcsname{\def\PY@tc##1{\textcolor[rgb]{0.73,0.13,0.13}{##1}}}
\expandafter\def\csname PY@tok@dl\endcsname{\def\PY@tc##1{\textcolor[rgb]{0.73,0.13,0.13}{##1}}}
\expandafter\def\csname PY@tok@s2\endcsname{\def\PY@tc##1{\textcolor[rgb]{0.73,0.13,0.13}{##1}}}
\expandafter\def\csname PY@tok@sh\endcsname{\def\PY@tc##1{\textcolor[rgb]{0.73,0.13,0.13}{##1}}}
\expandafter\def\csname PY@tok@s1\endcsname{\def\PY@tc##1{\textcolor[rgb]{0.73,0.13,0.13}{##1}}}
\expandafter\def\csname PY@tok@mb\endcsname{\def\PY@tc##1{\textcolor[rgb]{0.40,0.40,0.40}{##1}}}
\expandafter\def\csname PY@tok@mf\endcsname{\def\PY@tc##1{\textcolor[rgb]{0.40,0.40,0.40}{##1}}}
\expandafter\def\csname PY@tok@mh\endcsname{\def\PY@tc##1{\textcolor[rgb]{0.40,0.40,0.40}{##1}}}
\expandafter\def\csname PY@tok@mi\endcsname{\def\PY@tc##1{\textcolor[rgb]{0.40,0.40,0.40}{##1}}}
\expandafter\def\csname PY@tok@il\endcsname{\def\PY@tc##1{\textcolor[rgb]{0.40,0.40,0.40}{##1}}}
\expandafter\def\csname PY@tok@mo\endcsname{\def\PY@tc##1{\textcolor[rgb]{0.40,0.40,0.40}{##1}}}
\expandafter\def\csname PY@tok@ch\endcsname{\let\PY@it=\textit\def\PY@tc##1{\textcolor[rgb]{0.25,0.50,0.50}{##1}}}
\expandafter\def\csname PY@tok@cm\endcsname{\let\PY@it=\textit\def\PY@tc##1{\textcolor[rgb]{0.25,0.50,0.50}{##1}}}
\expandafter\def\csname PY@tok@cpf\endcsname{\let\PY@it=\textit\def\PY@tc##1{\textcolor[rgb]{0.25,0.50,0.50}{##1}}}
\expandafter\def\csname PY@tok@c1\endcsname{\let\PY@it=\textit\def\PY@tc##1{\textcolor[rgb]{0.25,0.50,0.50}{##1}}}
\expandafter\def\csname PY@tok@cs\endcsname{\let\PY@it=\textit\def\PY@tc##1{\textcolor[rgb]{0.25,0.50,0.50}{##1}}}

\def\PYZbs{\char`\\}
\def\PYZus{\char`\_}
\def\PYZob{\char`\{}
\def\PYZcb{\char`\}}
\def\PYZca{\char`\^}
\def\PYZam{\char`\&}
\def\PYZlt{\char`\<}
\def\PYZgt{\char`\>}
\def\PYZsh{\char`\#}
\def\PYZpc{\char`\%}
\def\PYZdl{\char`\$}
\def\PYZhy{\char`\-}
\def\PYZsq{\char`\'}
\def\PYZdq{\char`\"}
\def\PYZti{\char`\~}
% for compatibility with earlier versions
\def\PYZat{@}
\def\PYZlb{[}
\def\PYZrb{]}
\makeatother


    % Exact colors from NB
    \definecolor{incolor}{rgb}{0.0, 0.0, 0.5}
    \definecolor{outcolor}{rgb}{0.545, 0.0, 0.0}



    
    % Prevent overflowing lines due to hard-to-break entities
    \sloppy 
    % Setup hyperref package
    \hypersetup{
      breaklinks=true,  % so long urls are correctly broken across lines
      colorlinks=true,
      urlcolor=urlcolor,
      linkcolor=linkcolor,
      citecolor=citecolor,
      }
    % Slightly bigger margins than the latex defaults
    
    \geometry{verbose,tmargin=1in,bmargin=1in,lmargin=1in,rmargin=1in}
    
    

    \begin{document}
    
    
    \maketitle
    
    

    
    \subsection{\texorpdfstring{\textbf{Assignment 1: Information Retrieval
In High Dimensional
Data}}{Assignment 1: Information Retrieval In High Dimensional Data}}\label{assignment-1-information-retrieval-in-high-dimensional-data}

\subsubsection{Group 7:}\label{group-7}

\begin{itemize}
\tightlist
\item
  Verma, Shivendu (03692438)
\item
  Ben Chaabene, Siwar (03687915)
\item
  Mahajan, Shalvi (03708741)
\item
  Najib, Amna (03689277)
\item
  Sharma, Akshat (03708955)
\end{itemize}

    

    \textbf{Task1:} \textbf{Curse of Dimensionality }

**Let \$ C\_d = \{x \in \mathbb{R}\^{}p , \textbar{} , \{\lVert x
\rVert\}\_\infty \leq \frac{d}{2}\} \$ denote the p-dimensional
hypercube of edge length d, centered at the origin. \textbf{ }- Assume
\(X\) to be uniformly distributed in \(C_1\). Determine d in dependence
of pand \$ q \in [0, 1]\$, such that \[Pr(X \in C_d) = q\]
holds.\textbf{ }- Let the components of the p-dimensional random
variable \(X^p\) be independent and have the standard normal
distribution. It is known that \(Pr(|X^1| \leq 2.576) = 0.99\). For an
arbitrary p, determine the probability
\(Pr({\lVert x \rVert}_\infty > 2.576)\) for any of the components of
\(X^p\) to lie outside of the interval \([−2.576, 2.576]\). Evaluate the
value for \(p = 2\), \(p = 3\) and \(p = 500\).**

    \subsubsection{Answer: Task1}\label{answer-task1}

\paragraph{Part 1}\label{part-1}

    X is uniformly distributed in \(C_1\). Considering the lebesgue integral
and measure \(\lambda\)(\(C_1\)), the probability density function is
then expressed as follows for every \(x \in \mathbb{R}^p\):
\[ \begin{split} 
    f(x) & = \frac{1}{\lambda(C_1)} * \phi_{C_1}(x) \\
    \lambda(C_1) & = (\frac{1}{2}- \frac{-1}{2})^p \\
                 & = 1
    \end{split}  \]

, where \(\phi_{C_1}\) is the indicator function on the set \(C_1\). \$x
\in \mathbb{R}\^{}p \$ \[ 
\begin {align*} 
& Pr(X \in C_d) & = q \\
\Leftrightarrow \, \,& \int_{C_d} f(x) \, dx & = q \\
\Leftrightarrow \, \,& (\frac{d}{2}- \frac{-d}{2})^p & = q \\
\Leftrightarrow \, \,& d^p & = q \\
\Leftrightarrow \, \,& d & = \sqrt[p]{q}
\end{align*}
\]

    \paragraph{Part 2}\label{part-2}

    \[
\begin{split}
Pr({\lVert X^p \rVert}_\infty > 2.576) & = 1- Pr({\lVert X^p \rVert}_\infty \leq 2.576) \\
& = 1- Pr(|X^1| \leq 2.576 \, \, and \, \, |X^2| \leq 2.576 \, \, and \, \, .. \, \, and  \, \, |X^p| \leq 2.576) \\
& = 1- Pr(|X^1| \leq 2.576) * Pr(|X^2| \leq 2.576) * Pr(|X^3| \leq 2.576) * ... *  Pr(|X^p| \leq 2.576), \, \text{since the components are independent random variables}  \\
& = 1 -0.99 ^p
\end{split}
\]

    \begin{Verbatim}[commandchars=\\\{\}]
{\color{incolor}In [{\color{incolor}3}]:} \PY{k+kn}{import} \PY{n+nn}{numpy} \PY{k}{as} \PY{n+nn}{np}
        \PY{c+c1}{\PYZsh{} For p = 2}
        \PY{n}{Pr\PYZus{}2} \PY{o}{=} \PY{l+m+mi}{1}\PY{o}{\PYZhy{}} \PY{p}{(}\PY{n}{np}\PY{o}{.}\PY{n}{power}\PY{p}{(}\PY{l+m+mf}{0.99}\PY{p}{,}\PY{l+m+mi}{2}\PY{p}{)}\PY{p}{)}
        \PY{n+nb}{print}\PY{p}{(}\PY{l+s+s1}{\PYZsq{}}\PY{l+s+s1}{The probability value for p=2 }\PY{l+s+s1}{\PYZsq{}}\PY{p}{,} \PY{n}{Pr\PYZus{}2}\PY{p}{)}
        \PY{c+c1}{\PYZsh{} For p = 3}
        \PY{n}{Pr\PYZus{}3} \PY{o}{=} \PY{l+m+mi}{1}\PY{o}{\PYZhy{}} \PY{p}{(}\PY{n}{np}\PY{o}{.}\PY{n}{power}\PY{p}{(}\PY{l+m+mf}{0.99}\PY{p}{,}\PY{l+m+mi}{3}\PY{p}{)}\PY{p}{)}
        \PY{n+nb}{print}\PY{p}{(}\PY{l+s+s1}{\PYZsq{}}\PY{l+s+s1}{The probability value for p=3 }\PY{l+s+s1}{\PYZsq{}}\PY{p}{,} \PY{n}{Pr\PYZus{}3}\PY{p}{)}
        \PY{c+c1}{\PYZsh{} For p = 500}
        \PY{n}{Pr\PYZus{}500} \PY{o}{=} \PY{l+m+mi}{1}\PY{o}{\PYZhy{}} \PY{p}{(}\PY{n}{np}\PY{o}{.}\PY{n}{power}\PY{p}{(}\PY{l+m+mf}{0.99}\PY{p}{,}\PY{l+m+mi}{500}\PY{p}{)}\PY{p}{)}
        \PY{n+nb}{print}\PY{p}{(}\PY{l+s+s1}{\PYZsq{}}\PY{l+s+s1}{The probability value for p=500 }\PY{l+s+s1}{\PYZsq{}}\PY{p}{,} \PY{n}{Pr\PYZus{}500}\PY{p}{)}
\end{Verbatim}


    \begin{Verbatim}[commandchars=\\\{\}]
The probability value for p=2  0.01990000000000003
The probability value for p=3  0.029700999999999977
The probability value for p=500  0.9934295169575854

    \end{Verbatim}

    \textbf{Task 2: {[}10 Points{]}}

\textbf{Provide the PYTHON code to the following tasks (the code needs
to be commented properly):}

\begin{itemize}
\tightlist
\item
  \textbf{Sample 100 uniformly distributed random vectors from the box
  \([-1,1]^d\) for d=2}
\end{itemize}

    \begin{Verbatim}[commandchars=\\\{\}]
{\color{incolor}In [{\color{incolor}2}]:} \PY{k+kn}{import} \PY{n+nn}{numpy} \PY{k}{as} \PY{n+nn}{np}
        \PY{k+kn}{import} \PY{n+nn}{pandas} \PY{k}{as} \PY{n+nn}{pd}
        \PY{k+kn}{from} \PY{n+nn}{decimal} \PY{k}{import} \PY{o}{*}
        \PY{k+kn}{import} \PY{n+nn}{statistics}
        \PY{k+kn}{from} \PY{n+nn}{matplotlib} \PY{k}{import} \PY{n}{pyplot} \PY{k}{as} \PY{n}{plt}
        \PY{k+kn}{import} \PY{n+nn}{timeit}
        
        \PY{c+c1}{\PYZsh{}Function to sample (no\PYZus{}vec,dim) dimensional uniformly distributed random vectors between [\PYZhy{}1,1]}
        \PY{k}{def} \PY{n+nf}{generate\PYZus{}rv}\PY{p}{(}\PY{n}{no\PYZus{}vec}\PY{p}{,} \PY{n}{dim}\PY{p}{)}\PY{p}{:} 
            \PY{n}{rv} \PY{o}{=} \PY{n}{np}\PY{o}{.}\PY{n}{random}\PY{o}{.}\PY{n}{uniform}\PY{p}{(}\PY{o}{\PYZhy{}}\PY{l+m+mi}{1}\PY{p}{,} \PY{l+m+mi}{1}\PY{p}{,} \PY{p}{(}\PY{n}{no\PYZus{}vec}\PY{p}{,}\PY{n}{dim}\PY{p}{)}\PY{p}{)}
            \PY{k}{return} \PY{n}{rv}
        
        \PY{c+c1}{\PYZsh{}Mean Minimum angle to other vectors for 100 vectors}
        \PY{n}{dist}\PY{o}{=}\PY{n}{generate\PYZus{}rv}\PY{p}{(}\PY{l+m+mi}{100}\PY{p}{,}\PY{l+m+mi}{2}\PY{p}{)}
        \PY{n+nb}{print}\PY{p}{(}\PY{l+s+s2}{\PYZdq{}}\PY{l+s+s2}{100 uniformly distributed random vectors:}\PY{l+s+s2}{\PYZdq{}}\PY{p}{)}
        \PY{n+nb}{print}\PY{p}{(}\PY{n}{dist}\PY{p}{)}
\end{Verbatim}


    \begin{Verbatim}[commandchars=\\\{\}]
100 uniformly distributed random vectors:
[[-0.06971302  0.13133017]
 [ 0.64713025 -0.16409291]
 [ 0.43900646 -0.78895346]
 [-0.17159051  0.77990679]
 [-0.99906441 -0.01821633]
 [-0.95923275  0.61621158]
 [ 0.24470129  0.02614271]
 [-0.75209462  0.35326244]
 [ 0.76962764 -0.9471209 ]
 [ 0.7211867   0.01367141]
 [-0.8886856  -0.89467015]
 [-0.10398226 -0.82012708]
 [ 0.86484066  0.32991388]
 [-0.57684974  0.83989343]
 [-0.52582053  0.37611154]
 [ 0.50466665  0.71564939]
 [-0.82666635 -0.43747929]
 [ 0.94830878  0.83159537]
 [ 0.37776001  0.87628871]
 [-0.5267882  -0.36526441]
 [-0.74405449 -0.5898477 ]
 [-0.29219969  0.33508303]
 [ 0.74452436  0.14601904]
 [ 0.9752091   0.53170654]
 [ 0.33353369  0.25615258]
 [ 0.74353181 -0.15708582]
 [ 0.37185538  0.76885845]
 [ 0.24612533 -0.81707829]
 [-0.53628264 -0.90234315]
 [ 0.32292899 -0.96240896]
 [ 0.52874148  0.59207149]
 [ 0.5321894  -0.14266849]
 [-0.84513663 -0.23688465]
 [-0.10998966 -0.52032042]
 [-0.09098417  0.58930172]
 [-0.94915444 -0.05644647]
 [-0.62532787  0.53307772]
 [-0.62150864  0.11049048]
 [-0.64596585 -0.53295613]
 [-0.05929669 -0.98745413]
 [-0.10144154 -0.54736273]
 [-0.23680188  0.6424948 ]
 [ 0.52232015  0.01136041]
 [ 0.66206601 -0.4049458 ]
 [-0.65013209  0.86184243]
 [ 0.87997187 -0.32985814]
 [-0.40989448  0.64151205]
 [ 0.50074091 -0.32042057]
 [ 0.20095639 -0.1938596 ]
 [-0.7117915  -0.34435213]
 [ 0.90340883 -0.52588248]
 [-0.53148777  0.24140702]
 [ 0.72762018  0.62214022]
 [-0.28892657 -0.20542437]
 [ 0.98033505 -0.28068187]
 [-0.6977574   0.89963349]
 [-0.25370041 -0.67687043]
 [-0.07643233 -0.41448995]
 [ 0.00318403 -0.97697299]
 [-0.98308802  0.76087009]
 [ 0.6668667   0.43493494]
 [-0.56961851 -0.10999965]
 [-0.49410897 -0.53256992]
 [-0.68883968  0.08408443]
 [ 0.87830292 -0.98271065]
 [-0.77911295 -0.8226728 ]
 [-0.29341665  0.95220903]
 [ 0.07697369  0.10411203]
 [-0.9938065   0.46824516]
 [ 0.96827216  0.64785014]
 [ 0.45249035 -0.5544855 ]
 [-0.11058033  0.47757059]
 [-0.48828216 -0.75451357]
 [ 0.21990098 -0.96542262]
 [-0.00168193 -0.13690347]
 [ 0.36748865  0.98252351]
 [-0.96731974 -0.80459497]
 [ 0.40026784  0.51428603]
 [-0.63329079  0.80854259]
 [ 0.95327081  0.31408423]
 [-0.51170676  0.00182008]
 [ 0.86571086 -0.13517095]
 [ 0.5129557  -0.18009028]
 [-0.02864802  0.5287976 ]
 [ 0.29059842 -0.45792926]
 [-0.55473063 -0.6289338 ]
 [ 0.55448829  0.57542502]
 [-0.20027503 -0.02465741]
 [-0.04842558  0.28482339]
 [-0.03916781  0.31493276]
 [ 0.44369396  0.53682299]
 [-0.3687567   0.31404183]
 [-0.87531362 -0.47786638]
 [ 0.30405072 -0.39152984]
 [-0.37310975  0.74402974]
 [ 0.06964117  0.31602724]
 [-0.28146633  0.03592157]
 [-0.51063374 -0.73045261]
 [-0.11244028  0.49799575]
 [-0.88783962  0.05468154]]

    \end{Verbatim}

    \begin{itemize}
\tightlist
\item
  \textbf{For each of the 100 vectors determine the minimum angle to all
  other vectors. Then compute the average of these minimum angles. Note
  that for two vectors x, y the cosine of the angle between the two
  vectors is defined as} \[
  \begin{equation}
  \cos(\angle (x,y))=\frac{\langle x,y\rangle}{ \| x \| \| y\|}
  \end{equation}
  \]
\end{itemize}

    \begin{Verbatim}[commandchars=\\\{\}]
{\color{incolor}In [{\color{incolor}3}]:} \PY{c+c1}{\PYZsh{}Function to caclucate angles between vectors}
        \PY{k}{def} \PY{n+nf}{cos\PYZus{}sim}\PY{p}{(}\PY{n}{v1}\PY{p}{,} \PY{n}{v2}\PY{p}{)}\PY{p}{:}
            \PY{n}{cos}\PY{o}{=}\PY{p}{(}\PY{n+nb}{float}\PY{p}{(}\PY{n}{np}\PY{o}{.}\PY{n}{dot}\PY{p}{(}\PY{n}{v1}\PY{p}{,}\PY{n}{v2}\PY{p}{)}\PY{p}{)}\PY{p}{)}\PY{o}{/}\PY{p}{(}\PY{n}{np}\PY{o}{.}\PY{n}{linalg}\PY{o}{.}\PY{n}{norm}\PY{p}{(}\PY{n}{v1}\PY{p}{)}\PY{o}{*}\PY{n}{np}\PY{o}{.}\PY{n}{linalg}\PY{o}{.}\PY{n}{norm}\PY{p}{(}\PY{n}{v2}\PY{p}{)}\PY{p}{)}
            \PY{n}{angle\PYZus{}rad}\PY{o}{=}\PY{n}{np}\PY{o}{.}\PY{n}{arccos}\PY{p}{(}\PY{n}{cos}\PY{p}{)}
            \PY{n}{angle\PYZus{}deg}\PY{o}{=}\PY{n}{np}\PY{o}{.}\PY{n}{rad2deg}\PY{p}{(}\PY{n}{angle\PYZus{}rad}\PY{p}{)}
            \PY{k}{return} \PY{n}{angle\PYZus{}deg}
        
        \PY{c+c1}{\PYZsh{}Function for Mean of minimum angles}
        \PY{k}{def} \PY{n+nf}{mean\PYZus{}min\PYZus{}angle}\PY{p}{(}\PY{n}{rand\PYZus{}distribution}\PY{p}{)}\PY{p}{:}
            \PY{n}{res}\PY{o}{=}\PY{n}{np}\PY{o}{.}\PY{n}{zeros}\PY{p}{(}\PY{p}{(}\PY{n+nb}{len}\PY{p}{(}\PY{n}{rand\PYZus{}distribution}\PY{p}{)}\PY{p}{)}\PY{p}{,}\PY{n}{dtype}\PY{o}{=}\PY{n+nb}{float}\PY{p}{)}
            \PY{n}{mat}\PY{o}{=}\PY{n}{np}\PY{o}{.}\PY{n}{zeros}\PY{p}{(}\PY{p}{(}\PY{n+nb}{len}\PY{p}{(}\PY{n}{rand\PYZus{}distribution}\PY{p}{)}\PY{p}{,}\PY{n+nb}{len}\PY{p}{(}\PY{n}{rand\PYZus{}distribution}\PY{p}{)}\PY{p}{)}\PY{p}{,}\PY{n}{dtype}\PY{o}{=}\PY{n+nb}{float}\PY{p}{)}
            \PY{n}{mn}\PY{o}{=}\PY{n+nb}{float}\PY{p}{(}\PY{l+m+mi}{0}\PY{p}{)}
          
            \PY{k}{for} \PY{n}{i} \PY{o+ow}{in} \PY{n+nb}{range}\PY{p}{(}\PY{l+m+mi}{0}\PY{p}{,}\PY{n+nb}{len}\PY{p}{(}\PY{n}{rand\PYZus{}distribution}\PY{p}{)}\PY{p}{)}\PY{p}{:}
                \PY{k}{for} \PY{n}{j} \PY{o+ow}{in} \PY{n+nb}{range}\PY{p}{(}\PY{l+m+mi}{0}\PY{p}{,}\PY{n+nb}{len}\PY{p}{(}\PY{n}{rand\PYZus{}distribution}\PY{p}{)}\PY{p}{)}\PY{p}{:}
                    \PY{c+c1}{\PYZsh{}Calculate angles with all vectors}
                    \PY{k}{if}\PY{p}{(}\PY{n}{i}\PY{o}{!=}\PY{n}{j}\PY{p}{)}\PY{p}{:}
                        \PY{n}{mat}\PY{p}{[}\PY{n}{i}\PY{p}{,}\PY{n}{j}\PY{p}{]}\PY{o}{=}\PY{n}{cos\PYZus{}sim}\PY{p}{(}\PY{n}{rand\PYZus{}distribution}\PY{p}{[}\PY{n}{i}\PY{p}{]}\PY{p}{,}\PY{n}{rand\PYZus{}distribution}\PY{p}{[}\PY{n}{j}\PY{p}{]}\PY{p}{)}
                    \PY{k}{else}\PY{p}{:}
                        \PY{n}{mat}\PY{p}{[}\PY{n}{i}\PY{p}{,}\PY{n}{j}\PY{p}{]}\PY{o}{=}\PY{l+m+mi}{0}
                \PY{c+c1}{\PYZsh{}Compile minimum angles from all vectors}
                \PY{n}{res}\PY{p}{[}\PY{n}{i}\PY{p}{]}\PY{o}{=}\PY{n+nb}{sorted}\PY{p}{(}\PY{n}{mat}\PY{p}{[}\PY{n}{i}\PY{p}{]}\PY{p}{)}\PY{p}{[}\PY{l+m+mi}{1}\PY{p}{]}
            \PY{c+c1}{\PYZsh{}Mean of minimum angles}
            \PY{k}{return} \PY{n}{statistics}\PY{o}{.}\PY{n}{mean}\PY{p}{(}\PY{n}{res}\PY{p}{[}\PY{p}{:}\PY{p}{]}\PY{p}{)}
        
        \PY{n+nb}{print}\PY{p}{(}\PY{l+s+s2}{\PYZdq{}}\PY{l+s+s2}{Mean Minimum angles of 100 uniformly distributed random vectors: }\PY{l+s+s2}{\PYZdq{}}\PY{p}{,} \PY{n}{mean\PYZus{}min\PYZus{}angle}\PY{p}{(}\PY{n}{dist}\PY{p}{)}\PY{p}{)}
\end{Verbatim}


    \begin{Verbatim}[commandchars=\\\{\}]
Mean Minimum angles of 100 uniformly distributed random vectors:  1.7991970774197645

    \end{Verbatim}

    \begin{itemize}
\tightlist
\item
  \textbf{Repeat the above for dimensions d = 1,....,1000 and use the
  results to plot the average minimum angle against the dimension.}
\end{itemize}

    \begin{Verbatim}[commandchars=\\\{\}]
{\color{incolor}In [{\color{incolor}4}]:} \PY{c+c1}{\PYZsh{}Function to Plot For min angles}
        \PY{k}{def} \PY{n+nf}{plotMinAngle}\PY{p}{(}\PY{n}{no\PYZus{}vec}\PY{p}{,} \PY{n}{dim}\PY{p}{)}\PY{p}{:}
            \PY{n}{plt}\PY{o}{.}\PY{n}{title}\PY{p}{(}\PY{l+s+s1}{\PYZsq{}}\PY{l+s+s1}{Mean Minimum Similarity for }\PY{l+s+si}{\PYZpc{}d}\PY{l+s+s1}{ Vectors and }\PY{l+s+si}{\PYZpc{}d}\PY{l+s+s1}{ Dimensions}\PY{l+s+s1}{\PYZsq{}} \PY{o}{\PYZpc{}} \PY{p}{(}\PY{n}{no\PYZus{}vec}\PY{p}{,}\PY{n}{dim}\PY{p}{)}\PY{p}{)}
            \PY{n}{plt}\PY{o}{.}\PY{n}{xlabel}\PY{p}{(}\PY{l+s+s1}{\PYZsq{}}\PY{l+s+s1}{Dimensions}\PY{l+s+s1}{\PYZsq{}}\PY{p}{)}
            \PY{n}{plt}\PY{o}{.}\PY{n}{ylabel}\PY{p}{(}\PY{l+s+s1}{\PYZsq{}}\PY{l+s+s1}{Mean of minimum angle similarity}\PY{l+s+s1}{\PYZsq{}}\PY{p}{)}\PY{p}{;}
            \PY{n}{plot\PYZus{}arr}\PY{o}{=}\PY{n}{np}\PY{o}{.}\PY{n}{empty}\PY{p}{(}\PY{p}{(}\PY{l+m+mi}{1}\PY{p}{)}\PY{p}{,}\PY{n}{dtype}\PY{o}{=}\PY{n+nb}{float}\PY{p}{)}
        
            \PY{c+c1}{\PYZsh{}code\PYZus{}to\PYZus{}test=\PYZdq{}\PYZdq{}\PYZdq{}}
            \PY{k}{for} \PY{n}{i} \PY{o+ow}{in} \PY{n+nb}{range}\PY{p}{(}\PY{l+m+mi}{1}\PY{p}{,}\PY{n}{dim}\PY{o}{+}\PY{l+m+mi}{1}\PY{p}{)}\PY{p}{:}
                \PY{k}{if}\PY{p}{(}\PY{n}{i}\PY{o}{==}\PY{l+m+mi}{1}\PY{p}{)}\PY{p}{:}
                    \PY{n}{plot\PYZus{}arr}\PY{o}{=}\PY{n}{mean\PYZus{}min\PYZus{}angle}\PY{p}{(}\PY{n}{generate\PYZus{}rv}\PY{p}{(}\PY{n}{no\PYZus{}vec}\PY{p}{,}\PY{n}{i}\PY{p}{)}\PY{p}{)}
                \PY{k}{else}\PY{p}{:}
                    \PY{n}{plot\PYZus{}arr}\PY{o}{=}\PY{n}{np}\PY{o}{.}\PY{n}{append}\PY{p}{(}\PY{n}{plot\PYZus{}arr}\PY{p}{,}\PY{n}{mean\PYZus{}min\PYZus{}angle}\PY{p}{(}\PY{n}{generate\PYZus{}rv}\PY{p}{(}\PY{n}{no\PYZus{}vec}\PY{p}{,}\PY{n}{i}\PY{p}{)}\PY{p}{)}\PY{p}{)}
        
            \PY{n}{plt}\PY{o}{.}\PY{n}{plot}\PY{p}{(}\PY{n}{plot\PYZus{}arr}\PY{p}{)}
            \PY{n}{plt}\PY{o}{.}\PY{n}{show}\PY{p}{(}\PY{p}{)}
            
        \PY{n}{plotMinAngle}\PY{p}{(}\PY{l+m+mi}{100}\PY{p}{,}\PY{l+m+mi}{1000}\PY{p}{)}
\end{Verbatim}


    \begin{center}
    \adjustimage{max size={0.9\linewidth}{0.9\paperheight}}{output_13_0.png}
    \end{center}
    { \hspace*{\fill} \\}
    
    \begin{itemize}
\tightlist
\item
  \textbf{Give an interpretation of the result. What conclusions can you
  draw for 2 randomly sampled vectors in a d-dimensional space?}
\end{itemize}

    The curve of 100 Random variables plotted in dimensions form 1 to 1000
shows that when the dimension 'd' increases, the average minimum angle
tends to cos 90° or cosine similarity of 0. Which means that as in high
dimensional space, correlation between the same vectors decreases.

    \begin{itemize}
\tightlist
\item
  \textbf{Does the result change if the sample size increases?}
\end{itemize}

    \begin{Verbatim}[commandchars=\\\{\}]
{\color{incolor}In [{\color{incolor}5}]:} \PY{c+c1}{\PYZsh{}Increasing number of variables}
        \PY{n}{plotMinAngle}\PY{p}{(}\PY{l+m+mi}{300}\PY{p}{,}\PY{l+m+mi}{1000}\PY{p}{)}
\end{Verbatim}


    \begin{center}
    \adjustimage{max size={0.9\linewidth}{0.9\paperheight}}{output_17_0.png}
    \end{center}
    { \hspace*{\fill} \\}
    
    No, the result does not change as the number of samples increases, and
the above result of decreasing cosine similarity with increasing
dimensions still holds true.

    \textbf{Statistical Decision Making} \textbf{Task 3: {[}10 Points{]}}

\textbf{Answer the following questions. All answers must be justified.}

Figure 1:

\begin{longtable}[]{@{}llll@{}}
\toprule
x / y & 1 & 2 & 3\tabularnewline
\midrule
\endhead
1 & 0.02 & 0.26 & 0.13\tabularnewline
2 & 0.4 & 0.14 & 0.05\tabularnewline
\bottomrule
\end{longtable}

    \begin{itemize}
\tightlist
\item
  \textbf{The numbers in Figure 1 show the probability of the respective
  event to happen (e.g. the probability for the event \(X=1\) and
  \(Y=1\) is \(0.02\)). Is this table a probability table? If so, why?}
\end{itemize}

    The table will be a probability table only if the sum of the individual
probabilities of all events is 1.

\[
\because\,\; \sum_Y \sum_X p_{X, Y}(x, y) = 0.4+0.14+0.05+0.02+0.26+0.13 = 1
\] Also, \[
\because\,\; p_{X, Y}(x, y) \geq 0 \quad \forall\; (x, y) \in \{\mathbf{X} \times \mathbf{Y}\}
\] Hence, Figure 1 is a probability table.

    \begin{itemize}
\tightlist
\item
  \textbf{Based on Figure 1 give the conditional expectation
  \(\mathbb{E}_{Y \mid X=2}[Y]\) and the probability of the event
  \(X=1\) under the condition that \(Y=3\).}
\end{itemize}

    Conditional expectation \(\mathbb{E}_{Y \mid X=2}[Y]\): \[
\begin{align*}
& \because \,\; \mathbb{E}_{Y \mid X=2}[Y] = \sum_Y Y\;P(Y \mid X=2) \\
& \Rightarrow \mathbb{E}_{Y \mid X=2}[Y] = \sum_Y Y\;\frac{P(Y \cap X=2)}{P(X=2)} \\
& \Rightarrow \mathbb{E}_{Y \mid X=2}[Y] = \frac{1}{P(X=2)}\; \sum_Y Y\; P(Y \cap X=2) \\
& \Rightarrow \mathbb{E}_{Y \mid X=2}[Y] = \frac{1}{\sum_Y P(X=2, Y)}\sum_Y Y\;P(Y \cap X=2) \\
& \Rightarrow \mathbb{E}_{Y \mid X=2}[Y] = \frac{1}{0.4+0.14+0.05}(1\times 0.4 + 2\times 0.14 + 3\times 0.05) \\
& \therefore \,\; \mathbb{E}_{Y \mid X=2}[Y] = \frac{0.83}{0.59} \approx 1.4068
\end{align*}
\]

Conditional probability \(P(X=1 \mid Y=3)\): \[
\begin{align*}
& \because \,\; P(X=1 \mid Y=3) = \frac{P(X=1 \cap Y=3)}{P(Y=3)} \\
& \Rightarrow P(X=1 \mid Y=3) = \frac{P(X=1 \cap Y=3)}{\sum_X P(X, Y=3)} \\
& \Rightarrow P(X=1 \mid Y=3) = \frac{0.13}{0.13 + 0.05} \\
& \therefore \,\; P(X=1 \mid Y=3) = \frac{0.13}{0.18} \approx 0.722
\end{align*}
\]

    \begin{itemize}
\tightlist
\item
  \textbf{Is the function p(x, y) given by} \[
  p(x, y) = \begin{cases}
  1 & \quad \text{for } 0\leq x\leq 1\text{, } 0\leq y\leq \frac{1}{2} \\
  0 & \quad \text{otherwise} \end{cases}
  \] \textbf{a joint density function for two random variables?}
\end{itemize}

    For the function to be joint density function for two random variables,
\[
\int_y \int_x p(x, y) =1
\] Let's check this for \(p(x, y)\) \[
\because \int_y \int_x p(x, y)\; \mathrm{d}x \mathrm{d}y = \int_0^{1/2} \int_0^1 1\; \mathrm{d}x \mathrm{d}y = \int_0^{1/2} x\mid_0^1\; \mathrm{d}y = \int_0^{1/2} 1\; \mathrm{d}y = y\mid_0^{1/2} = \frac{1}{2} \neq 1
\] Therefore, \(p(x, y)\) is not a density function.

    \begin{itemize}
\tightlist
\item
  \textbf{For two random variables X and Y the joint density function is
  given by \[
  p(x, y) = \begin{cases}
  2\mathit{e}^{-(x+y)} & \quad \text{for } 0\leq x\leq y\text{, } 0\leq y \\
  0 & \quad \text{otherwise.} \end{cases}
  \] }What are the marginal density functions for X and Y
  respectively?**
\end{itemize}

    Marginal density function for \(\mathbf{X}\) : \[
\begin{align*}
& \because \,\; p(x) = \int_x^\infty p(x, y)\; \mathrm{d}y\\
& \Rightarrow p(x) = \lim_{u\to\infty} \int_x^u p(x, y)\; \mathrm{d}y\\
& \Rightarrow p(x) = \lim_{u\to\infty} \int_x^u 2e^{-(x+y)}\; \mathrm{d}y\\
& \Rightarrow p(x) = 2e^{-x} \lim_{u\to\infty} \int_x^u e^{-y}\; \mathrm{d}y\\
& \Rightarrow p(x) = 2e^{-x} \lim_{u\to\infty} \big[-e^{-y}\big]_{y=x}^{y=u}\\
& \Rightarrow p(x) = 2e^{-x} \lim_{u\to\infty} \big[-e^{-u}+e^{-x}\big]\\
& \Rightarrow p(x) = 2e^{-x} \big[0+e^{-x}\big]\\
& \therefore \,\; p(x) = 2e^{-2x}\\
\end{align*}
\] Using the same for \(\mathbf{Y}\) : \[
\begin{align*}
& \because \,\; p(y) = \int_0^y p(x, y)\; \mathrm{d}x\\
& \Rightarrow p(y) = \int_0^y 2e^{-(x+y)}\; \mathrm{d}x\\
& \Rightarrow p(y) = 2e^{-y} \int_0^y e^{-x}\; \mathrm{d}x\\
& \Rightarrow p(y) = 2e^{-y} \big[ -e^{-x} \big]_{x=0}^{x=y}\\
& \Rightarrow p(y) = 2e^{-y} \big[ -e^{-y} + e^0 \big]\\
& \therefore \,\; p(y) = 2e^{-y} \big( 1 - e^{-y} \big)\\
\end{align*}
\]

    \begin{itemize}
\tightlist
\item
  \textbf{Let the joint density function of two random variables X and Y
  be given by} \[
  p(x, y) = \begin{cases}
  \frac{1}{15}(2x+4y) & \quad \text{for } 0\leq x\leq 3\text{, } 0\leq y\leq 1 \\
  0 & \quad \text{otherwise.} \end{cases}
  \] \textbf{Determine the probability for \(X\leq 2\) under the
  condition that \(Y = \frac{1}{2}\).}
\end{itemize}

    \[
\begin{align*}
& \because \,\; p(y) = \int_x p(x, y)\; \mathrm{d}x\\
& \Rightarrow p(y) = \int_0^3 p(x, y)\; \mathrm{d}x\\
& \Rightarrow p(y) = \int_0^3 \frac{1}{15}(2x+4y)\; \mathrm{d}x\\
& \Rightarrow p(y) = \frac{1}{15}\; (x^2+4yx)\mid_{x=0}^3\\
& \therefore \,\; p(y) = \frac{1}{5}\; (4y+3)\\
\end{align*}
\] Therefore, \[
\begin{align*}
& \because \,\; P(X\leq 2\mid Y=\frac{1}{2}) = \int_0^2 \frac{p(x, y)}{p(y)}\mid_{y=\frac{1}{2}}\; \mathrm{d}x\\
& \Rightarrow P(X\leq 2\mid Y=\frac{1}{2}) = \int_0^2 \frac{2x+4y}{12y+9}\mid_{y=\frac{1}{2}}\; \mathrm{d}x\\
& \Rightarrow P(X\leq 2\mid Y=\frac{1}{2}) = \int_0^2 \frac{2x+2}{15}\; \mathrm{d}x\\
& \Rightarrow P(X\leq 2\mid Y=\frac{1}{2}) = \frac{1}{15}\; (x^2+2x)\mid_0^2\\
& \therefore \,\; P(X\leq 2\mid Y=\frac{1}{2}) = \frac{8}{15} \approx 0.533
\end{align*}
\]

    \textbf{Task 4: {[}3 Points{]}}

\textbf{Show that the covariance matrix C of any random variable
\(X \in R^p\) is symmetric positive semidefinite, i.e. \(C = C^T\) and
\$x\^{}TCx \geq \$  0 for any covariance matrix \$C \in R \^{} \{p
\times p\} \$ and any \(X \in R^p\)}

    The definition of the covariance matrix of a random vector
\(X \in R^p\), \(m_{x}\) is the mean vector:

\(C = E[(X-X_{m})(X-X_{m})^T]\)

The covariance matrix is symmetric (\(C = C^T\)):

\(C^T = E[(X-X_{m})(X-X_{m})^T]^T = E[((X-X_{m})(X-X_{m})^T)^T] = E[((X-X_{m})^T)^T(X-X_{m})^T] = E[((X-X_{m})(X-X_{m})^T)] = C\)

The covariance matrix is positive semidefinite ( \$ \forall  b
\in R\^{}p \$ , \(b^T C b \geq 0\)):

\$E{[}{[}(X-X\_\{m\})\^{}T b{]}\^{}2{]} = E{[}{[}(X-X\_\{m\})\^{}T
b{]}\^{}T{[}(X-X\_\{m\})\^{}T b{]}{]} \geq 0 \$, \$b \in R\^{}p \$

\(\Leftrightarrow\) \(E[b^T (X-X_{m})(X-X_{m})^T b] \geq 0\) , \$b
\in R\^{}p \$

\(\Leftrightarrow\) \(b^T C b \geq 0\) , \$b \in R\^{}p \$


    % Add a bibliography block to the postdoc
    
    
    
    \end{document}
